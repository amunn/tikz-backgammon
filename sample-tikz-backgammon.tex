% This is a test document for the tikz-backgammon package.
\documentclass[12pt]{article}
\usepackage{tikz-backgammon}
\makeatletter
\newcommand{\move}[1]{\bk@parsemove(#1)}
\def\bk@parsemove(#1/#2,#3/#4){\blackmove{#1}{#2}{#3}{#4}}
\makeatother
\begin{document}
\newgame
\blackboard
\move{24/18,24/17}
%\newgame
%\blankboard
%\blackpoint{1}{2}
\blackboard

\blackboard
\boardcaption{Black View}
\move{24/18,18/13}
\blackboard
\move{13/7,13/9}
\blackboard
\move{7/5,9/5}
\blackboard

\blackroll{66}
\blackboard

Black view
\blackmove*{24}{18}{13}{7}
\double{black}{2}
White view

\whiteboard
\whitemove*{6}{4}{13}{11}
\whiteroll{64}
\double{white}{4}
Black view
\whitepoint{23}{6}
\blackpoint{6}{8}
\boardcaption{A good position\\for all sorts of moves}
\blackboard
\whiteboard
\end{document}
\blackmove*{6}{5}{13}{9}
\whitemove*{19}{21}{17}{22}
\blackmove{8}{6}{8}{6}
\blackboard
\whitebo
%\blackmove{7}{6}{7}{6}
%
%\onbar{black}{2}
%\double{white}{2}
%\onbar{white}{2}
%\displayboard
\end{document}

